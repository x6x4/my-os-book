\documentclass[14pt]{extarticle}
\usepackage{extsizes}
\usepackage{geometry}
\geometry{margin=0.5in}

%% for images
\usepackage{graphicx}
\graphicspath{ {images/} }

%% language support
\usepackage[T1,T2A]{fontenc}
\usepackage[utf8]{inputenc}
\usepackage[english]{babel}

\usepackage{amsmath}
\usepackage{tikz}

%% hyperrefs
\usepackage{hyperref}
\hypersetup{
    colorlinks,
    citecolor=black,
    filecolor=black,
    linkcolor=black,
    urlcolor=magenta
}

%% some exotic
\setcounter{section}{-1}

\title{Operating systems and their parts}
\author{Qunix}
\date{01.01.1970}

\begin{document}
\maketitle
\tableofcontents

\part{Intro}
Operating systems are like fractals. The deeper you go, the more you 
are forced to know. Anyway, there's nothing more pleasant than reaching 
the real end and realizing you are at the very bottom 
of the unimaginable abyss.........
\\\\
However, we follow KISS! It means no Windows, no exotic, no challenges.
Just make things work and go ahead. 

\part{A Little OS}

When we try to do something, we always hope that it'll be simple to achieve
the result.
But for some awful reasons authors of most guides make monstrous algorithms
that are ultimately completely unusable.\\
To escape these problems, you should remember one thing - 
\texttt{OS is a program}. And the program needs the proper 
environment to be executed in. No environment - no execution!
\\
Please, do not touch your real computer for now. Our first task is to 
simulate the execution, not to play with disks. It has too much theory
behind. For now. We will return back to this ASAP.
\\
Starting with sandbox - guested isolated RE - we choose QEMU. 
A bit tricky to play with it freely, but fits excellent to copy $\&$ paste 
or change certain parameters. 

\section{Adventure begins}
The command \texttt{man qemu} tells us: 
\begin{verbatim}
    qemu-system-x86_64 [options] [disk_image]
\end{verbatim}
OK, we need disk image to be started. And here comes the great and 
terrible truth - the main feature of OS, distinguishing them from 
other programs, is their ability to be booted. BOOOOOOOtable image. 
It is achieved relatively easily - your program executable should 
end up with magic number 0x55aa and have 512 bytes in size.\\
At first, we make OS, which only hangs. Then, change register values 
and observe it with debugger. After this we discover the interrupts.
\\
As you could notice, we work with x86-64 architecture. It means reversed
byte order so don't be surprised.
\newpage
\section{Assembly}
We use nasm. \texttt{\$\$} - the beginning of our code, \texttt{\$} - the 
address of current line. 
\begin{verbatim}
    jmp $                          ;  infinite loop
    times 512 - 2 - ($-$$) db 0    ;  zeroing everything between
    dw 0xaa55                      ;  w - word = 2 bytes 
\end{verbatim}
Let's build it up to executable.\\
\texttt{nasm <input file> [-f <format>] [-o <output file>]}\\
You can write just \texttt{nasm myOS.asm} and you'll here. 
\texttt{hexdump myOS} says:
\begin{verbatim}
    0000000 feeb 0000 0000 0000 0000 0000 0000 0000
    0000010 0000 0000 0000 0000 0000 0000 0000 0000
    *
    00001f0 0000 0000 0000 0000 0000 0000 0000 aa55
    0000200
\end{verbatim}
Little-endian, as always, makes shit. You can see magic number, 
the total executable size (0x200 $= 2*16^2 = 512$), and also EB FE. 
According to \href{http://sparksandflames.com/files/x86InstructionChart.html}
{Intel x86 opcodes}, EB is a short (within a byte) jump and 
its argument, FE, is -2. 
Jump two bytes back is to start over. Infinite loop as it is. 
\\\\
The next step is to clutter up the registers. Just put up cluttering 
instruction (e.g. \texttt{mov eax, 555}) BEFORE loop, and you're done. 
\texttt{info registers} on qemu monitor should give you what you want to see. 
Another option is to use GDB. Add option \texttt{-s} to QEMU:\\
\texttt{-s Shorthand for -gdb tcp::1234, i.e. open a gdbserver 
on TCP port 1234}\\
And when QEMU starts, start gdb and type\\
\texttt{target remote localhost:1234}\\
Now debugger and your OS are connected, so you can, for example, 
clutter up registers in runtime. Use \texttt{set \$eax=0xdead} and
\texttt{print /x \$eax}. 

\section{Acquire hardware}
Guess you've played enough with registers. Doesn't make any sense, really?
Well, to do and see real things we need to know some more. 
BIOS interrupts allow us to do some basic things, but they are too 
non-visual while relatively hard to tell about. Skipping this step, we come 
to modifying peripherals - maybe video memory device. Don't you like 
colourful text? 

\section{Getting an upgrade}



\part{The Parts}
\section{Filesystems}
This history is prosaic and too applicative. FS is not OS, and therefore
there is a simple mechanism of adding new FS in Linux kernel. 
Read code, write code, execute code. No "AAAAAAAAAARGH MY HARDDISK HAS 
BEEN UNEXPECTEDLY REFORMATTED NOOOOO". Don't write OS, write FS. 
\\\\
How to write FS? We interpret it as "how to register new fs in linux kernel
and not to cry (as possible)". There are two chairs, and we use dynamic 
linking because most likely your laptop is a trash already and we don't 
want to clutter it up any further. So, the first part of question is 
"how to write a kernel module".   
\\
To write a kernel module, we just need a simple Makefile, which 
specifies our sources and then transfer control to a some of kernel 
Makefiles written exactly for making modules. Google this step yourself.
\\\\
As for the sources, there are also some rules of writing FS. You need 
to specify a chain of functions doing at least mounting and unmounting.
Then, you can check kernel log and see if it works. The working (for me)
example is in \href{https://github.com/x6x4/custom_fs}{my repo}.
\end{document}
