\documentclass[14pt]{extarticle}
\usepackage{extsizes}
\usepackage{geometry}
\geometry{margin=0.5in}

%% for images
\usepackage{graphicx}
\graphicspath{ {images/} }

%% language support
\usepackage[T1,T2A]{fontenc}
\usepackage[utf8]{inputenc}
\usepackage[english]{babel}

\usepackage{amsmath}
\usepackage{tikz}

%% hyperrefs
\usepackage{hyperref}
\hypersetup{
    colorlinks,
    citecolor=black,
    filecolor=black,
    linkcolor=black,
    urlcolor=cyan
}

\title{Operating systems and their parts}
\author{Qunix}
\date{01.01.1970}

\begin{document}
\maketitle
\tableofcontents

\part{Intro}
Operating systems are like fractals. The deeper you go, the more you 
are forced to know. Anyway, there's nothing more pleasant than reaching 
the real end and realizing you are at the very bottom of the unimaginable a(by)ss.........

\part{A Little OS}

\part{The Parts}
\section{Filesystems}
This history is prosaic and too applicative. FS is not OS, and therefore
there is a simple mechanism of adding new FS in Linux kernel. 
Read code, write code, execute code. No "AAAAAAAAAARGH MY HARDDISK HAS 
BEEN UNEXPECTEDLY REFORMATTED NOOOOO". Don't write OS, write FS. 
\\\\
How to write FS? We interpret it as "how to register new fs in linux kernel
and not to cry (as possible)". There are two chairs, and we use dynamic 
linking because most likely your laptop is a trash already and we don't 
want to clutter it up any further. So, the first part of question is 
"how to write a kernel module".   
\\
To write a kernel module, we just need a simple Makefile, which 
specifies our sources and then transfer control to a some of kernel 
Makefiles written exactly for making modules. Google this step yourself.
\\\\
As for the sources, there are also some rules of writing FS. You need 
to specify a chain of functions doing at least mounting and unmounting.
Then, you can check kernel log and see if it works. The working (for me)
example is in \href{https://github.com/x6x4/custom_fs}{my repo}.
\end{document}
